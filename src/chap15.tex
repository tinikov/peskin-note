\chapter{Non-Abelian Gauge Invariance}

% \nonumsec{Summary}
% \begin{itemize}
%   \item 占位
% \end{itemize}
% \pagestyle{general}

\section{The Geometry of Gauge Invariance}

\subsection{P483 - (15.5)}

此处的展开为
\begin{equation}\label{eq: expand_of_U(y,x)}
  U(x+\epsilon n, x) = 1 + \epsilon n^{\mu} \partial'_{\mu}U(x', x)|_{x' = x} + \mathcal{O}(\epsilon^2),
\end{equation}
我们将$ie^{-1}\partial'_{\mu}U(x', x)|_{x' = x}$记为$A_{\mu}$, 即可得到书上的结果.

\begin{mybox}{评论}
  此展开式也可以由$U(y, x)$更为一般的定义(参考15.3节)得到.
  我们令其起点为$0$, 终点为$s$, 则可记$U(s, 0)$为$U(s)$, 且$U(0) = 1$, 其定义式为
  \begin{equation}
    \frac{d}{ds}U(s) = i\frac{dx^{\mu}(s)}{ds}A_{\mu}[x(s)]U(s).
  \end{equation}

  \begin{equation}
    \begin{aligned}
      U(s+\epsilon) & = U(s+\epsilon)U(s)^{-1}U(s)                                                                      \\
                    & = \biggl[\biggl(U(s) + \epsilon\frac{d}{ds}U(s) \biggr)U^{-1}(s)\biggr]U(s)                       \\
                    & = \biggl[U(s)\biggl(1 + i\epsilon \frac{dx^{\mu}(s)}{ds}A_{\mu}[x(s)]\biggr)U^{-1}(s) \biggr]U(s) \\
                    & = \biggl(1 + i\epsilon \frac{dx^{\mu}(s)}{ds}A_{\mu}(x)\biggr)U(s),
    \end{aligned}
  \end{equation}
  令$s=0$, 则
  \begin{equation}
    U(\epsilon) = 1 + i\epsilon n^{\mu} A_{\mu}(x).
  \end{equation}

  这里的$A_{\mu}$和书中的差一个系数$-\frac{1}{e}$.
\end{mybox}

\subsection{P484 - (15.9)}

\begin{enumerate}
  \item 这里要求$(U(x,y))^{\dagger} = U(y,x)$是自然的.
        我们考虑对$(U(x,y))^{\dagger}$做书中(15.3)的变换, 有
        \begin{equation}
          (U(x,y))^{\dagger} \rightarrow (e^{i\alpha(x)}U(x,y)e^{-i\alpha(y)})^{\dagger} = e^{i\alpha(y)}(U(x,y))^{\dagger}e^{-i\alpha(x)},
        \end{equation}
        和$U(y,x)$的变换方式一致.

  \item 关于如何得到这个式子:

        首先我们将(\ref{eq: expand_of_U(y,x)})展开至$\epsilon^2$项(以下$\partial'_{\mu}U(x', x)|_{x' = x}$简写为$\partial_{\mu}U$)
        \begin{equation}
          \begin{aligned}
            U(x+\epsilon n, x) & = 1 + \epsilon n^{\mu} \partial_{\mu}U + \tfrac{1}{2}\epsilon^2 n^{\mu} n^{\nu} \partial_{\mu} \partial_{\nu}U + \mathcal{O}(\epsilon^3) \\
                               & = 1 - ie\epsilon n^{\mu}A_{\mu} - ie\epsilon n^{\mu} (\tfrac{1}{2} \epsilon n^{\nu} \partial_{\nu} A_{\mu}) + \mathcal{O}(\epsilon^3)    \\
                               & = 1 - ie\epsilon n^{\mu}l(A_{\mu} + \tfrac{1}{2} \epsilon n^{\nu} \partial_{\nu} A_{\mu}) + \mathcal{O}(\epsilon^3)                      \\
                               & = 1 - ie\epsilon n^{\mu}A_{\mu}(x+\tfrac{\epsilon}{2}n) + \mathcal{O}(\epsilon^3)                                                        \\
                               & = \exp[-ie\epsilon n^{\mu}A_{\mu}(x+\tfrac{\epsilon}{2}n) + \mathcal{O}(\epsilon^3)].
          \end{aligned}
        \end{equation}

  \item 检查这个式子满足$(U(x,y))^{\dagger} = U(y,x)$:
        \begin{equation}
          \begin{aligned}
            (U(x+\epsilon n, x))^{\dagger} & = \exp[+ie\epsilon n^{\mu}A_{\mu}(x+\tfrac{\epsilon}{2}n)]                \\
                                           & = \exp[+ie\epsilon n^{\mu}A_{\mu}((x+\epsilon n) - \tfrac{\epsilon}{2}n)] \\
                                           & = U((x+\epsilon n)-\epsilon n, x+\epsilon n)                              \\
                                           & = U(x, x+\epsilon n).
          \end{aligned}
        \end{equation}
\end{enumerate}

\subsection{P485 - (15.18)}

这里提到的dimension 5 and 6指mass dimension, 下面一行就有写.

\section{The Yang-Mills Lagrangian}

\subsection{P487 - (15.26)}

$V(x)V^{\dagger}(x) = 1$, 故$\partial_{\mu}(V(x)V^{\dagger}(x))=0$, 最后一行进行分部积分时用了这个性质.

\subsection{P488 - (15.30)}

对于有限变换, 证明如下:
\begin{equation}
  \begin{aligned}
    D_{\mu}\phi & = \biggl(\partial_{\mu} - igA^{i}_{\mu}\frac{\sigma^i}{2}\biggr)\psi                                                                  \\
                & \rightarrow \biggl(\partial_{\mu} - igV\biggl(A^{i}_{\mu}\frac{\sigma^i}{2} + \frac{i}{g}\partial_{\mu}\biggr)V^{\dagger}\biggr)V\psi \\
                & = \partial_{\mu}(V\psi) - VigA^{i}_{\mu}\frac{\sigma^i}{2}\psi + V(\partial_{\mu}V^{\dagger})V\psi                                    \\
                & = (\partial_{\mu}V)\psi + V\partial_{\mu}\psi - VigA^{i}_{\mu}\frac{\sigma^i}{2}\psi - (\partial_{\mu}V)V^{\dagger}V\psi              \\
                & = V\biggl(\partial_{\mu} - igA^{i}_{\mu}\frac{\sigma^i}{2}\biggr)\psi                                                                 \\
                & = VD_{\mu}\phi.
  \end{aligned}
\end{equation}

\subsection{P489 - (15.38)}

这里等式右端写成类似$-\frac{1}{4}\sum_{i}(F^i_{\mu\nu})^2$的形式会比较好理解.

\subsection{P491 - (15.50)下面那一段话}

\begin{mybox}{个人理解}
  参照书中(15.46)和(15.50), $\psi$和$F^a_{\mu\nu}$以及它们的协变导数的变换中都不含有$\partial_{\mu}\alpha^a$项($D_{\lambda}F^a_{\mu\nu}$的变换需要稍微算一下), 故全局对称性就可以保证局域对称性.
\end{mybox}

\section{The Gauge-Invariant Wilson Loop}

\subsection{P492 - (15.55)}

这里用斯托克斯定理后调整一下形式即可(第三行调整了$\epsilon$上标的顺序):
\begin{equation}
  \begin{aligned}
    U_P(y, y) & = \exp\Bigl[ -ie \oint_P dx^{\mu} A_{\mu}(x) \Bigr]                                                                                                                                                \\
              & = \exp\Bigl[ -ie \int_{\Sigma} \epsilon^{\mu\nu\rho\sigma}\partial_{\rho} A_{\sigma} dx_{\mu}dx_{\nu}\Bigr]                                                                                        \\
              & = \exp\Bigl[ -i\frac{e}{2} \int_{\Sigma} (\partial_{\rho} A_{\sigma} \epsilon^{\mu\nu\rho\sigma} dx_{\mu}dx_{\nu} - \partial_{\sigma} A_{\rho} \epsilon^{\mu\nu\rho\sigma} dx_{\mu}dx_{\nu})\Bigr] \\
              & = \exp\Bigl[ -i\frac{e}{2} \int_{\Sigma} F_{\rho\sigma}\epsilon^{\mu\nu\rho\sigma} dx_{\mu}dx_{\nu}\Bigr]                                                                                          \\
              & = \exp\Bigl[ -i\frac{e}{2} \int_{\Sigma} d\sigma^{\rho\sigma} F_{\rho\sigma} \Bigr].
  \end{aligned}
\end{equation}

\section{Basic Facts about Lie Algebras}

本节内容几乎是纯数学, 可以先只记结论.
如对本节感到困惑, 建议参考Brian C. Hall的Lie Groups, Lie Algebras, and Representations (GTM 222).

\begin{mybox}{吹牛建议}
  以后出门可以跟人吹高深莫测的牛: “我最近在看GTM 222”.
\end{mybox}

