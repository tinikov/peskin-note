\chapter{Elementary Processes of Quantum Electrodynamics}

\nonumsec{Summary}
\begin{itemize}
  \item 占位
\end{itemize}

\pagestyle{general}

\section{\texorpdfstring{$e^+e^- \rightarrow \mu^+\mu^-$}:: Introduction}

\subsection{P132 - (5.2)上面那个式子}

由于$\mathcal{M}$只是个\textit{c-number}, 并非矩阵, 故$\mathcal{M}^\dagger = \mathcal{M}^*$

\subsection{P132 - (5.3)上面那个求平均}

考虑入射粒子流由$n$种不同的粒子组成, 发现每种粒子的概率为$p_n(\sum p_n = 1)$, 那么总截面即为每种粒子的分截面的总和, 即
\begin{equation}
  |\mathcal{M}|^2 = \sum p_n |\mathcal{M}_n|^2,
\end{equation}
书中的例子考虑自旋向上或向下的概率均为$1/2$, 即完全非极化的情况.

由于入射流的粒子情况是已知的, 粒子总数目固定, 每种粒子所占的比例已知, 所以求总截面时计算各分截面的期望; 而对于出射流的粒子, 由于探测器忽略了自旋信息, 故需对所有携带有不同自旋信息的事件直接求和(否则结果中含有的自旋信息探测器探测不到).

\subsection{P132 - (5.4)}
\begin{mybox}{矩阵的trace作为内积}
  题外话: 对矩阵的乘积取trace是一种常见的在由矩阵构成的线性空间中定义内积的方法.
  简单算一下会发现, 这个内积就是两个矩阵所有对应的元的乘积的和.
\end{mybox}

\subsection{P134 - (5.6)}

由于只有当$\alpha\beta\gamma\delta$均不相同时$\epsilon^{\alpha\beta\gamma\delta}$才不为$0$, 故相乘时的4个度规必取$g^{00}g^{11}g^{22}g^{33}=-1$.
这就是结果中负号的来源.

\section{\texorpdfstring{$e^+e^- \rightarrow \mu^+\mu^-$}:: Helicity Structure}

\subsection{P142 - (5.18)下面一段话}

前面的\textit{right}-handed spinner $v(p')$指这个旋量$v(p')$是螺度算符的右手本征态$\psi_R$; 而后面的\textit{left}-handed positron指由$b^\dagger_{\mathbf{p}'}$算符产生的态(与此旋量$v(p')$相关)在自旋算符的作用下有本征值$-\frac{1}{2}$. (参考书中P61最下方那一大段话)

\subsection{P144 - Figure 5.4}

将粒子运动方向选为$z$轴的正方向, 考虑\textit{right}-handed旋量.
由于无质量粒子的螺度本征值不改变, 则正粒子的自旋($J_z$)必定为$\frac{1}{2}$, 反粒子的自旋必定为$-\frac{1}{2}$.
图中两电子的运动方向相反, 于是当选定电子$e^-$的运动方向为$z$轴的正方向时, 正电子的自旋自然也就变成了$\frac{1}{2}$.

\subsection{P144 - (5.26)}

参考笔记中的 \eqref{eq: spinor_explicit(ch.3)} 式或是Appendix中的 \nameref{appendix:spinor_field} 一节中旋量的具体表达式可以很简单地得到这个形式.

\section{\texorpdfstring{$e^+e^- \rightarrow \mu^+\mu^-$}:: Nonrelativistic limit}

\subsection{P146 - (5.33)}

$E\rightarrow m_{\mu}$即$|\mathbf{k}|\rightarrow 0$时, 可将书中(5.12)式改写为
\begin{equation}
  \frac{d\sigma}{d\Omega}=\frac{\alpha^2}{4E^2_{\rm cm}}\frac{|\mathbf{k}|}{E}\biggl[2+ \frac{|\mathbf{k}|}{E}^2 \cos^2\theta\biggr],
\end{equation}
保留关于$|\mathbf{k}|$的一阶项, 即为书中的结果.
